\documentclass{report}
\usepackage[utf8]{inputenc}
\usepackage[russian]{babel}
\author{Иҷрокунанда: Садриддинзода Некрузҷон }

\begin{document}

\title{ТАҲИЯИ ПОРТАЛИ КУРСИ ОНЛАЙН БО ИСТИФОДА АЗ СИСТЕМАҲОИ ТАВСИЯДИҲАНДА ДАР АСОСИ АЛГОРИТМҲОИ ОМӮЗИШИ МОШИНӢ }
\maketitle 

Таҳияи портали курси онлайн бо истифода аз системаҳои тавсиядиҳанда дар асоси алгоритмҳои омӯзиши мошинсозӣ якчанд марҳилаҳоро дар бар мегирад. Дар ин ҷо нақшаи умумии раванд оварда шудааст: 
 
 
 
           \textbf{Муайян кардани мушкилот:}  Қадами аввал ин муайян кардани, мушкилоти таҳияи портали курсҳои онлайн аст, ки курсҳоро ба корбарон дар асоси манфиатҳои онҳо ва рафтори гузашта тавсия медиҳад. 
 
           \textbf{Ҷамъоварии маълумот:} Барои сохтани системаи тавсиядиҳанда маълумот дар бораи корбарон ва рафтори онҳо лозим аст. Ин маълумот метавонад аз сарчашмаҳои гуногун, аз ҷумла профилҳои корбар, каталогҳои курсҳо ва ҳамкории корбарон бо портал гирифташавад. 
 
           \textbf{Алгоритми омӯзиши мошинӣ.} Якчанд алгоритмҳои омӯзиши мошинсозӣ мавҷуданд, ки онҳоро барои сохтани системаи тавсиядиҳанда истифода бурдан мумкин аст, аз ҷумла филтркунии муштарак, филтр дар асоси мундариҷа ва равишҳои гибридӣ. 
 
          \textbf{Омӯзонидани модел:} Пас аз интихоби алгоритм тақсимоти маълумотро ба маҷмӯаҳои омӯзишӣ ва санҷишӣ, мувофиқ кардани модел ба маълумоти омӯзишӣ ва арзёбии иҷрои он дар маълумоти санҷишӣ дар бар мегирад. 
 
         Таҳияи портали курсҳои онлайн бо истифода аз системаҳои тавсиядиҳанда дар асоси алгоритмҳои омӯзиши мошинсозӣ истифодаи якчанд технологияҳоро талаб мекунад. Инҳоянд баъзе аз технологияҳои асосӣ: 
 
       \textbf{Забонҳои барномасозӣ:} Маъмул барои таҳияи веб Python, JavaScript ва PHP мебошанд. 
 
       \textbf{Чаҳорчӯбаҳои таҳияи веб:} Чаҳорчӯбаҳои маъмули таҳияи веб Flask, Django мебошанд. 
 
       \textbf{Технологияҳои пойгоҳи додаҳо:} Технологияҳои маъмули пойгоҳи додаҳо MySQL,      MongoDB ва PostgreSQL мебошанд. 
 
       \textbf{Китобхонаҳои омӯзиши мошинҳо:} Китобхонаҳои маъмул барои омӯзиши мошинсозӣ дар Python Scikit-learn, TensorFlow ва PyTorch мебошанд. 
 
         \textbf{Платформаҳои роёниши абрӣ:} Платформаҳои маъмули роёниши абрӣ Amazon Web Services (AWS), Google Cloud Platform (GCP) ва Microsoft Azure мебошанд. 
 
         Технологияҳои беруна: Технологияҳои маъмултарини фронталӣ HTML, CSS ва JavaScript, инчунин чаҳорчӯба ба монанди React, Angular ва Vue.js мебошанд. 
         
\end{document}